%%%%%%%%%%%%%%%%%%%%%%%%%%%%%%%%%%%%%%%%%%%%%%%%%%%%%%%%%%%%%%%%%%%%%%%%
%%%%%%%%%%%%%%%%%%%%%% Simple LaTeX CV Template %%%%%%%%%%%%%%%%%%%%%%%%
%%%%%%%%%%%%%%%%%%%%%%%%%%%%%%%%%%%%%%%%%%%%%%%%%%%%%%%%%%%%%%%%%%%%%%%%

%%%%%%%%%%%%%%%%%%%%%%%%%%%% Document Setup %%%%%%%%%%%%%%%%%%%%%%%%%%%%

\documentclass[10pt]{article}

\usepackage[utf8]{inputenc}
\usepackage[brazil]{babel}

% The automated optical recognition software used to digitize resume
% information works best with fonts that do not have serifs. This
% command uses a sans serif font throughout. Uncomment both lines (or at
% least the second) to restore a Roman font (i.e., a font with serifs).
\usepackage{times}
\renewcommand{\familydefault}{\sfdefault}

% The OCR software also has a hard time with italics. These commands get
% rid of the two common ways to italicize text in LaTeX. Get rid of them
% to turn italics back on.
\renewcommand\emph[1]{#1}
\renewcommand\textit[1]{#1}

% This is a helpful package that puts math inside length specifications
\usepackage{calc}

% Layout: Puts the section titles on left side of page
\reversemarginpar

% Use these lines for A4-sized paper
\usepackage[paper=a4paper,
            %includefoot, % Uncomment to put page number above margin
            marginparwidth=30.5mm,    % Length of section titles
            marginparsep=1.5mm,       % Space between titles and text
            margin=25mm,              % 25mm margins
            includemp]{geometry}

%% More layout: Get rid of indenting throughout entire document
\setlength{\parindent}{0in}

\usepackage[shortlabels]{enumitem}


\usepackage{fancyhdr,lastpage}
\pagestyle{fancy}
\pagestyle{empty}      % Uncomment this to get rid of page numbers
\fancyhf{}\renewcommand{\headrulewidth}{0pt}
\fancyfootoffset{\marginparsep+\marginparwidth}
\newlength{\footpageshift}
\setlength{\footpageshift}
          {0.5\textwidth+0.5\marginparsep+0.5\marginparwidth-2in}

%%%% PAGES 2--9 NUMBERING:
%% These two lines put page number in upper-right corner of pages 2--end
\rhead{Amadeus Folego da Silva~\arabic{page} of \protect\pageref*{LastPage}}   % +LP
%\rhead{Pavlic, p.~\arabic{page}}                                 % -LP

%% These lines put page number in bottom (center) of pages 2--end
%\lfoot{\hspace{\footpageshift}%
%       \parbox{4in}{\, \hfill %
%                    \arabic{page} of \protect\pageref*{LastPage} % +LP
%%                    \arabic{page}                               % -LP
%                    \hfill \,}}
%%%% END PAGE 2--9 NUMBERING

%%%% PAGE 1 NUMBERING:
\makeatletter
\let\oldps@plain\ps@plain
\renewcommand{\ps@plain}{\oldps@plain%
\renewcommand{\@evenfoot}{\hfil %
    p.~\arabic{page} of \protect\pageref*{LastPage} % +LP
%    p.~\arabic{page}                               % -LP
    \hfil}%
\renewcommand{\@oddfoot}{\@evenfoot}}
\makeatother
%%%% END PAGE 1 NUMBERING

% Finally, give us PDF bookmarks and colored links
%
% NOTE: Some OCR software might be negatively affected by hyperlinks. So
%       most employers recommend the draft option here.
%
% (to enable hyperlinks and bookmarks, comment out ``draft'' line;
%  to disable hyperlinks and bookmarks, uncomment ``draft'' line)
\usepackage{color,hyperref}
\definecolor{darkblue}{rgb}{0.0,0.0,0.3}
\hypersetup{colorlinks,breaklinks,
            linkcolor=darkblue,urlcolor=darkblue,
            anchorcolor=darkblue,citecolor=darkblue,
            %draft
            }

%%%%%%%%%%%%%%%%%%%%%%%% End Document Setup %%%%%%%%%%%%%%%%%%%%%%%%%%%%


%%%%%%%%%%%%%%%%%%%%%%%%%%% Helper Commands %%%%%%%%%%%%%%%%%%%%%%%%%%%%

\newcommand{\makeheading}[2][]%
        {\hspace*{-\marginparsep minus \marginparwidth}%
         \begin{minipage}[t]{\textwidth+\marginparwidth+\marginparsep}%
             {\large \bfseries #2 \hfill #1}\\[-0.15\baselineskip]%
                 \rule{\columnwidth}{1pt}%
         \end{minipage}}

\renewcommand{\section}[1]{\pagebreak[3]%
    \hyphenpenalty=10000%
    \vspace{1.3\baselineskip}%
    \phantomsection\addcontentsline{toc}{section}{#1}%
    \noindent\llap{\scshape\smash{\parbox[t]{\marginparwidth}{\raggedright #1}}}%
    \vspace{-\baselineskip}\par}

\newenvironment{outerlist}[1][\enskip\textbullet]%
        {\begin{itemize}[#1,leftmargin=*]}{\end{itemize}%
         \vspace{-.6\baselineskip}}

\newenvironment{innerlist}[1][\enskip\textbullet]%
        {\begin{itemize}[#1,leftmargin=*,parsep=0pt,itemsep=0pt,topsep=0pt,partopsep=0pt]}
        {\end{itemize}}

% To add some paragraph space between lines.
% This also tells LaTeX to preferably break a page on one of these gaps
% if there is a needed pagebreak nearby.
\newcommand{\blankline}{\quad\pagebreak[3]}
\newcommand{\halfblankline}{\quad\vspace{-0.5\baselineskip}\pagebreak[3]}

% Uses hyperref to link github and twitter
\newcommand\github[1]{\href{http://github.com/#1}{@#1}}
\newcommand\twitter[1]{\href{http://twitter.com/#1}{@#1}}

% For \url{SOME_URL}, links SOME_URL to the url SOME_URL
\providecommand*\url[1]{\href{#1}{#1}}
% Same as above, but pretty-prints SOME_URL in teletype fixed-width font
\renewcommand*\url[1]{\href{#1}{\texttt{#1}}}

% For \email{ADDRESS}, links ADDRESS to the url mailto:ADDRESS
\providecommand*\email[1]{\href{mailto:#1}{#1}}
% Same as above, but pretty-prints ADDRESS in teletype fixed-width font
\renewcommand*\email[1]{\href{mailto:#1}{\texttt{#1}}}

%%%%%%%%%%%%%%%%%%%%%%%% End Helper Commands %%%%%%%%%%%%%%%%%%%%%%%%%%%

%%%%%%%%%%%%%%%%%%%%%%%%% Begin CV Document %%%%%%%%%%%%%%%%%%%%%%%%%%%%

\begin{document}

\makeheading[Curriculum vitae]{Amadeus Folego}

\vspace{1em}
\begin{center}

  \hspace{-10em} {\Large AMADEUS FOLEGO DA SILVA}\\[.5em]
  \hspace{-10em} {Natural de Mauá, SP, 23/12/1988 - 1 filho - Solteiro}\\
  \hspace{-10em} {Rua Aurora, 787, apto. 52, Centro. São Paulo - SP, 01209-001}\\
  \hspace{-10em} {}

\end{center}

\section{Informações para Contato}

\newlength{\rcollength}\setlength{\rcollength}{2.5in}

\begin{tabular}[t]{@{}p{\textwidth-\rcollength}p{\rcollength}}
 Email: \email{amadeusfolego@gmail.com}   & Celular: 11 9554-3190 \\
        Github: \github{badosu} & Twitter: \twitter{badosu\_}\\

\end{tabular}

\section{Perfil}

  Profissional versátil com experiência em desenvolvimento para a web, utilizando principalmente .Net e Ruby.
  Interessando em Matemática, Ciências da Computação, Engenharia e {\em Hacking} em geral.\\
  Possui familiaridade e boa capacidade para aprendizado de conceitos e tecnologias relevantes para a web.

\section{Experiência Profissional}
  \vspace{3em}
  \begin{innerlist}{\labelwidth 18em \leftmargin 1.53in \itemsep -.025in }
    \item [\textbf{01/2012-05/2012}]{\textbf{Boo-box}~~~\scriptsize \url{http://boo-box.com} \normalsize}\\
      Empresa de tecnologia, onde tive contato com sistemas de boa escala.
    \item [\textbf{06/2010 - 11/2011}]{\textbf{Betboo}~~~\scriptsize \url{http://betboo.com} \normalsize}\\
      Site de jogos eletrônicos, onde tive a oportunidade de trabalhar em projetos de expansão para Europa
    \item [\textbf{11/2009 - 06/2010}]{\textbf{Lumina1}}\\
      Agência digital onde trabalhei em diversos projetos e iniciei minha experiência com Web
  \end{innerlist}

\section{Projetos Interessantes}
  \vspace{3em}
  \begin{innerlist}{\labelwidth 1em \leftmargin 1.53in \itemsep -.025in }
    \item{\textbf{Borel}~~~\scriptsize \url{http://rubydoc.info/gems/borel/} \normalsize}\\
      Criei essa gem para resolver alguns problemas recorrentes em projetos que exigiam algoritmos em conjuntos ordenados.
    \item{\textbf{Audi Quattro para Ipad}~~~\scriptsize \url{http://vimeo.com/36426085} \normalsize}\\
      Desenvolvi um framework para mapeamento de elementos HTML em um modelo de física bi-dimensional, utilizando extensivamente javascript e css-transforms.
  \end{innerlist}


\section{Qualificações Acadêmicas}
  \vspace{3em}
  \begin{innerlist}{\labelwidth 18em \leftmargin 1.13in \itemsep -.025in }
    \item[\textbf{01/2010 - 12/2012}]
      {\bf Bacharelado em Matemática} Universidade Federal do ABC (previsto)
    \item[\textbf{09/2006 - 12/2010}]
      {\textbf{Bacharelado em Ciência e Tecnologia}} Universidade Federal do ABC
      \begin{itemize}
        \item Modelagem Dinâmica do Fenômeno de Histerese Magnética\\
          {\scriptsize Disponível em: \url{http://www.scribd.com/doc/21709933}}
        \item {\bf Monitoria Acadêmica}
          \begin{innerlist}
            \item Funções de Várias Variáveis
            \item Cálculo Vetorial e Tensorial
          \end{innerlist}
      \end{itemize}
    \item [\bf 01/2004 - 06/2005] {\bf Técnico em Informática} Escola Técnica Estadual Jorge Street
  \end{innerlist}

\end{document}

%%%%%%%%%%%%%%%%%%%%%%%%%% End CV Document %%%%%%%%%%%%%%%%%%%%%%%%%%%%%

%----------------------------------------------------------------------%
% The following is copyright and licensing information for
% redistribution of this LaTeX source code; it also includes a liability
% statement. If this source code is not being redistributed to others,
% it may be omitted. It has no effect on the function of the above code.
%----------------------------------------------------------------------%
% Copyright (c) 2007, 2008, 2009, 2010, 2011 by Theodore P. Pavlic
%
% Unless otherwise expressly stated, this work is licensed under the
% Creative Commons Attribution-Noncommercial 3.0 United States License. To
% view a copy of this license, visit
% http://creativecommons.org/licenses/by-nc/3.0/us/ or send a letter to
% Creative Commons, 171 Second Street, Suite 300, San Francisco,
% California, 94105, USA.
%
% THE SOFTWARE IS PROVIDED "AS IS", WITHOUT WARRANTY OF ANY KIND, EXPRESS
% OR IMPLIED, INCLUDING BUT NOT LIMITED TO THE WARRANTIES OF
% MERCHANTABILITY, FITNESS FOR A PARTICULAR PURPOSE AND NONINFRINGEMENT.
% IN NO EVENT SHALL THE AUTHORS OR COPYRIGHT HOLDERS BE LIABLE FOR ANY
% CLAIM, DAMAGES OR OTHER LIABILITY, WHETHER IN AN ACTION OF CONTRACT,
% TORT OR OTHERWISE, ARISING FROM, OUT OF OR IN CONNECTION WITH THE
% SOFTWARE OR THE USE OR OTHER DEALINGS IN THE SOFTWARE.
%----------------------------------------------------------------------%
